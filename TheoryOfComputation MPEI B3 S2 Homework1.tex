\documentclass{article}
\usepackage[utf8]{inputenc}
\usepackage[russian]{babel}
\usepackage[pdf]{graphviz}
%\usepackage[pdftex]{graphicx}
%\usepackage{graphviz}
%\usepackage{mathtools}

\title{Теоретические модели вычислений \\ ДЗ №1: Регулярные языки и конечные автоматы}
\author{Студент группы А-13а-19 Башлыков М.С.}
\date{April 2022}

\begin{document}

\pagenumbering{gobble}
\maketitle
\newpage
\pagenumbering{arabic}

\section{Задание №1. Построить конечный автомат,
распознающий язык}
\begin{enumerate}
    \item \(L = \{\omega \in \{a,b,c\}^* | |\omega|_c = 1 \} \) \\ \\
        Так как нам нужен лишь один символ ``c``, то просто будем ждать прочтения этого символа. Если же нам встретится ещё один символ ``c``, покинем конечное состояние:
    \begin{center}
    %\begin{figure}
    %\centering
    %\def\svgwidth{\columnwidth}
    %\input{DFA11.svg}
    %\end{figure}
    %\includegraphics[scale=0.4]{DFA11.svg}
    \digraph[scale=1]{DFA11}{
        rankdir = "LR";
        q0 [shape=circle];
        q1 [shape=doublecircle];
        q2 [shape=circle];
        q0 -> q0 [label = "a, b"];
        q0 -> q1 [label = "c"];
        q1 -> q1 [label = "a, b"];
        q1 -> q2 [label = "c"];
        q2 -> q2 [label = "a, b, c"];
    }
    \end{center}
    
    \item \(L = \{\omega \in \{a,b\}^* | |\omega|_a \leq 2, \; |\omega|_b \geq 2 \} \) \\ \\
        Построим отдельно ДКА для \(L_1 = \{\omega \in \{a,b\}^* | |\omega|_a \leq 2 \} \) и для \(L_2 = \{\omega \in \{a,b\}^* | |\omega|_b \geq 2 \} \), а затем получим их прямое произведение: \\ \\
        $L_1$: Будем менять состояние при чтении $a$, а при чтении $b$ переходить в то же самое состояние. При этом начальное состояние и состояния при первых двух чтениях $a$ будут конечными, а уже третье считывание приведёт к непринимающему состоянию.
    \begin{center}
    \digraph[scale=1]{DFA121}{
        rankdir = "LR";
        q0 [shape=doublecircle];
        q1 [shape=doublecircle];
        q2 [shape=doublecircle];
        q3 [shape=circle];
        q0 -> q0 [label = "b"];
        q0 -> q1 [label = "a"];
        q1 -> q1 [label = "b"];
        q1 -> q2 [label = "a"];
        q2 -> q2 [label = "b"];
        q2 -> q3 [label = "a"];
        q3 -> q3 [label = "a, b"];
    }
    \end{center}
        $L_2$: Будем аналогично менять состояния при считывании символа $b$ и входить в то же самое состояние при считывании $a$. Однако здесь непринимающими будут состояния до считывания $b$ и после первого считывания символа $b$, второй же символ $b$ приведёт нас в конечное состояние.
    \begin{center}
    \digraph[scale=1]{DFA122}{
        rankdir = "LR";
        q4 [shape=circle];
        q5 [shape=circle];
        q6 [shape=doublecircle];
        q4 -> q4 [label = "a"];
        q4 -> q5 [label = "b"];
        q5 -> q5 [label = "a"];
        q5 -> q6 [label = "b"];
        q6 -> q6 [label = "a, b"];
    }
    \end{center}
        Тогда итоговый ДКА будет являться прямым произведением:
    \begin{center}
    \digraph[scale=0.5]{DFA12}{
        rankdir = "LR";
        q04 [shape=circle];
        q05 [shape=circle];
        q06 [shape=doublecircle];
        q14 [shape=circle];
        q15 [shape=circle];
        q16 [shape=doublecircle];
        q24 [shape=circle];
        q25 [shape=circle];
        q26 [shape=doublecircle];
        q34 [shape=circle];
        q35 [shape=circle];
        q36 [shape=circle];
        
        q04 -> q14 [label = "a"];
        q05 -> q15 [label = "a"];
        q06 -> q16 [label = "a"];
        q14 -> q24 [label = "a"];
        q15 -> q25 [label = "a"];
        q16 -> q26 [label = "a"];
        q24 -> q34 [label = "a"];
        q25 -> q35 [label = "a"];
        q26 -> q36 [label = "a"];
        q34 -> q34 [label = "a"];
        q35 -> q35 [label = "a"];
        
        q04 -> q05 [label = "b"];
        q05 -> q06 [label = "b"];
        q06 -> q06 [label = "b"];
        q14 -> q15 [label = "b"];
        q15 -> q16 [label = "b"];
        q16 -> q16 [label = "b"];
        q24 -> q25 [label = "b"];
        q25 -> q26 [label = "b"];
        q26 -> q26 [label = "b"];
        q34 -> q35 [label = "b"];
        q35 -> q36 [label = "b"];
        
        q36 -> q36 [label = "a, b"];
    }
    \end{center}
    
    \item \(L = \{\omega \in \{a,b\}^*||\omega|_a \neq |\omega|_b \}\) \\
        Это нерегулярный язык, а значит, для него нельзя построить ДКА. Докажем, применив лемму о разрастании:
    \begin{enumerate}
        \item Рассмотрим дополнение к языку: \(\neg L = \{\omega \in \{a,b\}^*||\omega|_a = |\omega|_b \}\).
        \item Пусть \(\neg L\) – регулярный. Тогда выполняется лемма о разрастании.
        \item Зафиксируем \(n \in N\). Пусть \(\omega = a^{n}b^{n}\). Тогда \(\omega \in \neg L\) и \(|\omega| \geq n\).
        \item Рассмотрим всевозможные разбиения слова \(\omega = xyz\) такие, что \(|xy| \leq n\) и \(y \neq \lambda\):\\
        \(x = a^{l}, 0 \leq l \leq n - 1\)\\
        \(y = a^{m}, 1 \leq m \leq n - l\)\\
        \(z = a^{n - l - m}b^{n}\)\\
        Тогда при \(k = 2\) \(xy^{k}z \notin \neg L\), т.к. тогда \(xy^{k}z = xy^{2}z = a^{l + 2m + n - l - m}b^{n} = a^{n + m}b^{n}\) и \(n + m \neq m\), т.е. для слова \(xy^{k}z\) не выполяется условие \(|\omega|_a = |\omega|_b\). Следовательно, лемма о разрастании не выполняется.\\
        Т.к. для дополнения языка не выполняется лемма о разрастании, то оно не является регулярным, следовательно, и изначальный язык не является регулярным.
    \end{enumerate}
        
    
    \item \(L = \{\omega \in \{a,b\}^* | \omega \omega = \omega \omega \omega \} \) \\
        Исходя из определения этого языка, если $|\omega| = x$, то получается, что $2x = |\omega \omega| = |\omega \omega \omega| = 3x$, следовательно, $x = 0$, т.е. ДКА должен допускать лишь пустые слова:
    \begin{center}
    \digraph[scale=1]{DFA14}{
        rankdir = "LR";
        q0 [shape=doublecircle];
        q1 [shape=circle];
        q0 -> q1 [label = "a, b"];
        q1 -> q1 [label = "a, b"];
    }
    \end{center}
\end{enumerate}

\section{Задание №2. Построить конечный автомат,
используя прямое произведение}
\begin{enumerate}
    \item \(L_1 = \{\omega \in\{a,b\}^* | |\omega|_a \geq 2 \wedge |\omega|_b        \geq 2 \} \) \\
        Построим ДКА отдельно для каждого условия, а затем возьмём их прямое произведение. \\
        \(L_{11}= \{\omega \in\{a,b\}^* | |\omega|_a \geq 2 \} \):
        \begin{center}
            \digraph[scale=1]{DFA211}{
                rankdir = "LR";
                q0 [shape=circle];
                q1 [shape=circle];
                q2 [shape=doublecircle];
                q0 -> q0 [label = "b"];
                q0 -> q1 [label = "a"];
                q1 -> q1 [label = "b"];
                q1 -> q2 [label = "a"];
                q2 -> q2 [label = "a, b"];
            }
        \end{center}
        \(L_{12}= \{\omega \in\{a,b\}^* | |\omega|_b \geq 2 \} \):
        \begin{center}
            \digraph[scale=1]{DFA212}{
                rankdir = "LR";
                q3 [shape=circle];
                q4 [shape=circle];
                q5 [shape=doublecircle];
                q3 -> q3 [label = "a"];
                q3 -> q4 [label = "b"];
                q4 -> q4 [label = "a"];
                q4 -> q5 [label = "b"];
                q5 -> q5 [label = "a, b"];
            }
        \end{center}
        Прямое произведение:
        \begin{center}
            \digraph[scale=0.5]{DFA21}{
                rankdir = "LR";
                q03 [shape=circle];
                q04 [shape=circle];
                q05 [shape=circle];
                q13 [shape=circle];
                q14 [shape=circle];
                q15 [shape=circle];
                q23 [shape=circle];
                q24 [shape=circle];
                q25 [shape=doublecircle];
                
                q03 -> q13 [label = "a"];
                q04 -> q14 [label = "a"];
                q05 -> q15 [label = "a"];
                q13 -> q23 [label = "a"];
                q14 -> q24 [label = "a"];
                q15 -> q25 [label = "a"];
                q23 -> q23 [label = "a"];
                q24 -> q24 [label = "a"];
        
                q03 -> q04 [label = "b"];
                q04 -> q05 [label = "b"];
                q05 -> q05 [label = "b"];
                q13 -> q14 [label = "b"];
                q14 -> q15 [label = "b"];
                q15 -> q15 [label = "b"];
                q23 -> q24 [label = "b"];
                q24 -> q25 [label = "b"];
                
                q25 -> q25 [label = "a, b"];
        
            }
        \end{center}
        
    \begin{center}
    \end{center}
    
    \item \(L_2 = \{\omega \in\{a,b\}^* | |\omega|_a \geq 3 \wedge |\omega|_b
        {нечётное} \} \) \\
    \begin{center}
    \end{center}
    
    \item \(L_3 = \{\omega \in\{a,b\}^* | |\omega|_a {чётно} \wedge |\omega|_b
        {кратно трём} \} \) \\
    \begin{center}
    \end{center}
    
    \item \(L_4 = \neg L_3\) \\
        ДКА может быть получен аналогично п. {1.3}, но затем нужно инвертировать состояния:
    \begin{center}
    \end{center}
    
    \item \(L_5 = L_2 \setminus L_3\) \\
        Т.к. $A \setminus B = A \cap \neg B$, а $L_4 = \neg L_3$, то нам остаётся построить прямое произведение ДКА для $L_2$ и $L_4$
    \begin{center}
    \end{center}
\end{enumerate}

\section{Задание №3. Построить минимальный ДКА
по регулярному выражению}
\begin{enumerate}
    \item \((ab + aba)^{*}a\) \\
    
    \begin{center}
    \end{center}
    
    \item \(a(a(ab)^{*}b)^{*}(ab)^{*}\) \\
    
    \begin{center}
    \end{center}
    
    \item \((a + (a + b)(a + b)b)^*\) \\
    
    \begin{center}
    \end{center}
    
    \item \((b+c)((ab)^*c + (ba)^*)^*\) \\
    
    \begin{center}
    \end{center}
    
    \item \((a+b)^+(aa + abab + bb + baba)(a+b)^+\) \\
    
    \begin{center}
    \end{center}
\end{enumerate}

\section{Задание №4. Определить, является ли язык
регулярным или нет}
\begin{enumerate}
    \item \(L = \{(aab)^{n}b(aba)^{m} | n \geq 0, \; m \geq 0\}\) \\
        Язык является регулярным, т.к. можно построить ДКА, распознающий его:
    \begin{center}
    \digraph[scale=0.5]{DFA41}{
        rankdir = "LR";
        q0 [shape=circle];
        q1 [shape=circle];
        q2 [shape=circle];
        q3 [shape=circle];
        q4 [shape=doublecircle];
        q5 [shape=circle];
        q6 [shape=circle];
        q0 -> q1 [label = "a"];
        q0 -> q4 [label = "b"];
        q1 -> q2 [label = "a"];
        q2 -> q3 [label = "b"];
        q3 -> q1 [label = "a"];
        q3 -> q4 [label = "b"];
        q4 -> q5 [label = "a"];
        q5 -> q6 [label = "b"];
        q6 -> q4 [label = "a"];
    }
    \end{center}
    
    \item \(L = \{uaav | u \in \{a, b\}^*, \; v \in \{a, b\}^*, \; |u|_b \geq         |v|_a\}\) \\
        Язык не является регулярным. Докажем, применив лемму о разрастании:
    \begin{enumerate}
        \item Рассмотрим дополнение к языку: \(\neg L\).
        \item Пусть \(\neg L\) – регулярный. Тогда выполняется лемма о разрастании.
        \item Зафиксируем \(n \in N\). Пусть \(\omega = b^{2n}aaa^{2n + 1}\). Тогда \(\omega \in \neg L\) и \(|\omega| \geq n\).
        \item Рассмотрим всевозможные разбиения слова \(\omega = xyz\) такие, что \(|xy| \leq n\) и \(y \neq \lambda\):\\
        \(x = b^{l}, 0 \leq l \leq n - 1\)\\
        \(y = b^{m}, 1 \leq m \leq n - l\)\\
        \(z = b^{2n - l - m}aaa^{2n + 1}\)\\
        Тогда при \(k = 2\) \(xy^{k}z \notin \neg L\), т.к. тогда \(xy^{k}z = xy^{2}z = b^{l + 2m + 2n - l - m}aaa^{2n + 1} = b^{2n + m}aaa^{2n + 1}\) и \(2n + m \geq 2n + 1\), т.е. для слова \(xy^{k}z\) выполяется условие \(|u|_b \geq |v|_a\}\) и \(xy^{k}z \in L\). Следовательно, лемма о разрастании не выполняется.\\
        Т.к. для дополнения языка не выполняется лемма о разрастании, то оно не является регулярным, следовательно, и изначальный язык не является регулярным.
    \end{enumerate}
    
    \item \(L = \{a^{m}w | w \in \{a, b\}^{*}, \; 1 \leq |w|_b \leq m\}\) \\
        Язык не является регулярным. Докажем, применив лемму о разрастании:
    \begin{enumerate}
        \item Рассмотрим дополнение к языку: \(\neg L\).
        \item Пусть \(\neg L\) – регулярный. Тогда выполняется лемма о разрастании.
        \item Зафиксируем \(n \in N\). Пусть \(\omega' = a^{n}b^{n + 1}\). Тогда \(\omega' \in \neg L\) и \(|\omega'| \geq n\).
        \item Рассмотрим всевозможные разбиения слова \(\omega' = xyz\) такие, что \(|xy| \leq n\) и \(y \neq \lambda\):\\
        \(x = a^{l}, 0 \leq l \leq n - 1\)\\
        \(y = a^{m}, 1 \leq m \leq n - l\)\\
        \(z = a^{n - l - m}b^{n + 1}\)\\
        Тогда при \(k = 2\) \(xy^{k}z \notin \neg L\), т.к. тогда \(xy^{k}z = xy^{2}z = a^{l + 2m + n - l - m}b^{n + 1} = a^{n + m}b^{n + 1}\) и \(n + m \geq n + 1\), т.е. для слова \(xy^{k}z\) выполяется условие: \(\omega' = a^{n + m}b^{n + 1} = a^{n + m}\omega\) и \(1 \leq |\omega|_b \leq n + m\), т.е. \(\omega' \in L\). Следовательно, лемма о разрастании не выполняется.\\
        Т.к. для дополнения языка не выполняется лемма о разрастании, то оно не является регулярным, следовательно, и изначальный язык не является регулярным.
    \end{enumerate}
    
    \item \(L = \{a^{k}b^{m}a^{n} | k = n \vee m > 0\}\) \\
        Язык не является регулярным. Докажем, применив лемму о разрастании:
    \begin{enumerate}
        
    \end{enumerate}
    
    \item \(L = \{ucv | u \in \{a, b\}^*, \; v \in \{a, b\}^*, \; u \neq v^R \}\) \\
        Язык не является регулярным. Докажем, применив лемму о разрастании:
    \begin{enumerate}
        \item Рассмотрим дополнение к языку: \(\neg L\).
        \item Пусть \(\neg L\) – регулярный. Тогда выполняется лемма о разрастании.
        \item Зафиксируем \(n \in N\). Пусть \(\omega = a^{2n + 2}ca^{n + 1}\). Тогда \(\omega \in L\) и \(|\omega| \geq n\).
        \item Рассмотрим всевозможные разбиения слова \(\omega = xyz\) такие, что \(|xy| \leq n\) и \(y \neq \lambda\):\\
        \(x = a^{l}, 0 \leq l \leq n - 1\)\\
        \(y = a^{m}, 1 \leq m \leq n - l\)\\
        \(z = a^{2n + 2 - l - m}ca^{n + 1}\)\\
        Тогда при \(k = 2\) \(xy^{k}z \notin \neg L\), т.к. тогда \(xy^{k}z = xy^{2}z = a^{l + 2m + 2n + 2 - l - m}ca^{n + 1} = a^{2n + 2 + m}ca^{n + 1}\) и \(1 \leq m \leq n\), т.е. \(1 \leq (2n + 2 + m) mod (n + 1) \leq n\). Следовательно, для слова \(xy^{k}z\) выполяется условие: \(\omega = ucv | u \in \{a, b\}^*, \; v \in \{a, b\}^*, \; u \neq v^R \}\), т.е. \(\omega' \in L\). Следовательно, лемма о разрастании не выполняется.\\
        Т.к. для дополнения языка не выполняется лемма о разрастании, то оно не является регулярным, следовательно, и изначальный язык не является регулярным.
    \end{enumerate}
\end{enumerate}

\end{document}
